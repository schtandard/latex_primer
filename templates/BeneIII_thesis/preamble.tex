% !TeX root = thesis
% !TeX encoding = utf8
% !TeX spellcheck = de_DE
\documentclass[
  ngerman,
  twoside,
  titlepage=firstiscover,
  openright,
  captions=tableheading,
  BCOR=.5cm,
  fontsize=11,
  ]{scrreprt}

% The {fontspec} package allows font control with LuaTeX or XeTeX.
% The {unicode-math} package allows the use of unicode math fonts.
\usepackage{fontspec}
\usepackage[
  math-style=ISO,
  partial=upright,
  warnings-off={mathtools-overbracket, mathtools-colon},
  ]{unicode-math}

% The {etoolbox} package gives us some additional macro manipulation capabilities.
% The {xparse} package provides an advanced interface for defining commands (LaTeX3).
\usepackage{etoolbox}
\usepackage{xparse}

%% For general manipulation:

% The {xcolor} package provides advanced color capabilities.
\usepackage{xcolor}

% The {scalerel} package provides commands for arbitrary scaling of boxes.
\usepackage{scalerel}


%% For page layout:

% The {pdflscape} package allows typesetting parts of the document in landscape mode (pdf compatible).
% The {rotating} package provides several rotating macros and, notably, sideways floats.
%\usepackage{pdflscape}
%\usepackage[figuresright]{rotating}

% The {scrlayer-scrpage} package from the koma script bundle allows changing header and footer.
\pagestyle{headings}
\usepackage{scrlayer-scrpage}


%% For typesetting text:

% The {babel} package makes sure LaTeX knows your language (choose your language as the package loading option).
% The {microtype} package conducts "subliminal refinements towards typographical perfection", i.e. makes your document look even better.
% The {nowidow} package prevents widows and orphans.
\usepackage[
%  english,
  main=ngerman,
  ]{babel}
\usepackage{microtype}
\usepackage[
  defaultlines=2,
  all,
  ]{nowidow}

% The {csquotes} package provides context sensitive quoting facilities and is recommended by {biblatex}.
\usepackage{csquotes}

% The {array} package provides advanced table creation tools.
% The {booktabs} package provides tools to make better tables (most notably \toprule, \midrule and \bottomrule).
\usepackage{array}
\usepackage{booktabs}

% The {enumitem} package provides improved list environment capabilities.
%\usepackage{enumitem}


%% For typesetting maths:

% The {bm} package provides bold math symbols.
% The {mathtools} package provides more handy math commands, it is an extension package to {amsmath}.
% The {mleftright} package provides \mleft and \mright as alternatives to \left and \right with better spacing.
\usepackage{bm}
\usepackage{mathtools}
\usepackage{mleftright}

% The {dsfont} package provides double stroke math symbols.
%\usepackage{dsfont}


%% The {siunitx} package has to be loaded after the math packages.

% The {siunitx} package provides all commands needed to print numeric values and units, most notably the commands \num, \si and \SI as well as the [S] table column.
% The {mhchem} package provides commands for typesetting chemical formulas and related expressions.
% We need to define \xtimes now, because siunitx checks its existence (and possibly size).
\let\xtimes\times
\let\times\cdot
\usepackage[
  locale=DE,
  product-symbol=\xtimes,
  ]{siunitx}
\usepackage[version=4]{mhchem}


%% For images:

% The {graphicx} package provides an improved graphics interface.
% The {svg} package allows you to include svg images into your document (via \includesvg) using the capabilities of Inkscape (see also http://mirrors.ctan.org/info/svg-inkscape/InkscapePDFLaTeX.pdf).
% shell-escape (write18, respectively) must be enabled for this.
% \pdfsuppresswarningpagegroup=1 turns off a (usually) benign warning.
\usepackage{graphicx}
\graphicspath{{./img/}}
\usepackage[inkscapearea=page, usexcolor]{svg}
\svgpath{{./img/}}


%% For cross-referencing:

% The {biblatex} package provides advanced bilbiography capabilities.
% The {hyperref} package converts all internal references to links and offers a lot of further options.
% The {caption} package offers the capability to fix where Links to a figure lead (to the image instead of the caption) and removes the colon before the caption text when it is empty.
% The {subcaption} package provides commands for setting captions to subfloats.
\usepackage[style=phys]{biblatex}
\usepackage[
    hidelinks,
    pdfencoding=auto,
    bookmarksnumbered,
  ]{hyperref}
\usepackage[
    hypcap,
  ]{caption}
\usepackage{subcaption}


%%% symbols

%% just symbols
\makeatletter
  % constants
  \protected\def\e{{\symup e}}
  \protected\def\i{{\symup i}}
  \protected\def\cpi{{\uppi}}
  \sisetup{
    input-symbols=\e\i\cpi,
  }
  % for use in other macros
  \def\@difsym{{\symup d}}
  \def\@delsym{{\symup\partial}}
  \def\@Difsym{{\symup D}}
  \def\@DifFeynsym{{\symcal D}}
  \def\@bndrysym{{\symup\partial}}
\makeatother

% unary operators
\makeatletter
  \AtBeginDocument{%
    \DeclareMathOperator\grad{grad}
    \let\div\undefined
    \DeclareMathOperator\div{div}
    \DeclareMathOperator\rot{rot}
    \DeclareMathOperator\laplace{\Delta}
%    \DeclareMathOperator\laplace{\symup{\Delta}}
    % \dif[<exponent>]{<var>}
    % starred version for use at start of integral
    % double starred version for use at start of enumerator
    \NewDocumentCommand\dif{s s o m}{%
      \IfBooleanF{#2}{\IfBooleanTF{#1}{\unskip}{\mathop{}}\!}%
      \@difsym%
      \IfValueT{#3}{^{#3}}%
      #4%
      \IfBooleanT{#1}{\mathop{}\!}%
    }
    % \del[<exponent>]{<var>}
    % starred version adds no spaces
    \NewDocumentCommand\del{s o m}{%
      \IfBooleanF{#1}{\mathop{}\!}%
      \@delsym_{#3}%
      \IfValueT{#2}{^{#2}}%
    }
    \NewDocumentCommand\bndry{m}{{\@bndrysym #1}}
  }
\makeatother

% binary operators
\makeatletter
  \AtBeginDocument{%
    \let\xtimes\times
    \let\times\cdot
    \let\vectimes\xtimes
  }
\makeatother

% operators
\makeatletter
  % A command for typesetting expressions with (possibly scaled) delimiters.
  % Use as follows: \scaled@delimiter@expression[<scale>]{<left_delimiter>}{<middle_delimiter>}{<right_delimiter>}{<expression>}
  % Input \auto for auto-scaling, . for obsolete delimiters.
  % In the <expression>, type \<pos>@dlmtr where the delimiters belong (<pos> being left, middle or right).
  \NewDocumentCommand\scaled@delimiter@expression{o m m m m}{\begingroup
    \def\@auto##1{\middle##1\let\auto\@auto}%
    \def\left@dlmtr{\let\auto\mleft\ifx\auto#1\relax\else\mathopen\fi\IfValueT{#1}{#1}#2\let\auto\@auto}%
    \def\right@dlmtr{\let\auto\mright\ifx\auto#1\relax\else\mathclose\fi\IfValueT{#1}{#1}#4}%
    \def\middle@dlmtr{\IfValueT{#1}{#1}#3}%
    #5
  \endgroup}
  \NewDocumentCommand\abs{o m}{%
    \scaled@delimiter@expression[#1]{\lvert}{.}{\rvert}{\left@dlmtr#2\right@dlmtr}%
  }
  \NewDocumentCommand\mean{o m o}{%
    \scaled@delimiter@expression[#1]{\langle}{.}{\rangle}{\left@dlmtr#2\right@dlmtr}%
    \IfValueT{#3}{_{#3}}%
  }
\makeatother

% quotient-like operators
\makeatletter
  \NewDocumentCommand\d@quotient{m o m m}{%
    \frac{#1\IfValueT{#2}{^{#2}\!}#3}%
      {#1#4\IfValueT{#2}{^{#2}\!}}%
  }
  \NewDocumentCommand\dd@quotient{m m m m}{%
    \frac{#1^2#2}%
      {#1#3\ifx~#4\relax^{2}\!\else#1#4\fi}%
  }
  \newcommand*\diff{\d@quotient\@difsym}
  \newcommand*\ddiff{\dd@quotient\@difsym}
  \newcommand*\dell{\d@quotient\@delsym}
  \newcommand*\ddell{\dd@quotient\@delsym}
\makeatother

% Vector and matrix commands
\makeatletter
  \AtBeginDocument{%
    \let\arrow\vec
    \let\vec\symbf
  }
%  \def\uvec#1{\hat{\vec{#1}}}
  \newcommand*\vect[1]{\begin{pmatrix}#1\end{pmatrix}}
  \newcommand*\absvect[1]{\abs[\auto]{\vect{#1}}}
\makeatother

% operators to the right
\makeatletter
  \NewDocumentCommand\evalwith{O{\bigg} m}{\mathop{}\!#1|_{#2}}
  \NewDocumentCommand\fromto{O{\bigg} m m}{\mathop{}\!#1|_{#2}^{#3}}
\makeatother



%%% macros

%% patches

% Center all floats by default.
\makeatletter
  \appto\@floatboxreset{\centering}
\makeatother

% Change the default style of lists.
\makeatletter
  % itemize
  \renewcommand*\labelitemi{{$\circ$}}
  \renewcommand*\labelitemii{{\raise.175ex\hbox{$\scriptstyle\diamond$}}}
  \renewcommand*\labelitemiii{{\raise.285ex\hbox{$\scriptscriptstyle\triangleright$}}}
  \renewcommand*\labelitemiv{{\raise.285ex\hbox{$\scaleobj{.5}\ast$}}}% needs {scalerel}
  %\renewcommand*\labelitemiv{{\raise.375ex\hbox{$\scaleobj{.25}{\square}$}}}% needs {scalerel}
  %\renewcommand*\labelitemiv{{\raise.35ex\hbox{$\centerdot$}}}
  % enumerate
  \renewcommand*\theenumi{{\roman{enumi})}}
  \renewcommand*\theenumii{{\sffamily\small\arabic{enumii}}}
  \renewcommand*\theenumiii{{\sffamily\alph{enumiii}}}
  \renewcommand*\theenumiv{{\sffamily\small\Alph{enumiv}}}
  \renewcommand*\p@enumi{}
  \renewcommand*\p@enumii{\p@enumi\theenumi-}
  \renewcommand*\p@enumiii{\p@enumii\theenumii-}
  \renewcommand*\p@enumiv{\p@enumiii\theenumiii-}
  \renewcommand*\labelenumi{\theenumi}
  \renewcommand*\labelenumii{\theenumii}
  \renewcommand*\labelenumiii{\theenumiii}
  \renewcommand*\labelenumiv{\theenumiv}
  % description
  %\renewcommand*\descfont{\sffamily\bfseries}
\makeatother

% {hyperref}
\makeatletter
  % \url[no-print-prefix]{url}
  \begingroup
    \endlinechar=-1 %
    \catcode`\^^A=14 %
    \catcode`\^^M\active
    \catcode`\%\active
    \catcode`\#\active
    \catcode`\_\active
    \catcode`\$\active
    \catcode`\&\active
    \let\my@url\hyper@normalise
    \patchcmd\my@url{\hyper@n@rmalise}{\my@url@}{}{}^^A
    \global\let\my@url\my@url
  \endgroup
  \NewDocumentCommand\my@url@{O{} m}{\endgroup\href{#1#2}{\nolinkurl{#2}}}
  \renewcommand*\url{\my@url}
  % \email{e-mail-address}[display-text]
  \DeclareRobustCommand*\email{\hyper@normalise\my@email}
  \newcommand*\my@email[1]{\my@email@{#1}}
  \NewDocumentCommand\my@email@{m O{\nolinkurl{#1}}}{\href{mailto:#1}{#2}}
\makeatother

% {array}
% Add aligned fixed width table column specifiers
% Usage: L{<style>}{<width>} with <style> one of p, m, b (C and R analogously).
\makeatletter
  \newcolumntype{L}[2]{>{\raggedright\let\newline\\\arraybackslash\strut}#1{#2}}
  \newcolumntype{C}[2]{>{\centering\let\newline\\\arraybackslash\strut}#1{#2}}
  \newcolumntype{R}[2]{>{\raggedleft\let\newline\\\arraybackslash\strut}#1{#2}}
\makeatother

% {subcaption}
\makeatletter
  % Make subfloats center by default.
  \apptocmd\subcaption@iiminipage{\centering}{}{}
\makeatother

% {rotating}
%\makeatletter
%  % Make sideways floats be a minipage with fixed height, so that \vfill will work.
%  \patchcmd\@xrotfloat{\begin{minipage}\textheight}{\begin{minipage}[c][\textwidth][c]\textheight}{}{}
%  \patchcmd\end@rotfloat{\wd\rot@float@box\z@\ht\rot@float@box\z@\dp\rot@float@box\z@}{}{}{}
%  \def\endsidewaysfigure{\end@rotfloat}
%  \def\endsidewaystable{\end@rotfloat}
%  % Make sideways floats center by default.
%  \apptocmd\@xrotfloat{\centering}{}{}
%  % Rotate pages with sideways floats (using the mechanism created below).
%  \apptocmd\@xrotfloat{\thispagepdflscape}{}{}
%  % Provide a length that contains the \textwidth outside of a sidewaysfigure (i.e. the \textheight when rotated).
%  \newdimen\outertextwidth
%  \pretocmd\@xrotfloat{\outertextwidth=\textwidth}{}{}
%\makeatother


%% miscellaneous macros

\makeatletter
  % \wrap{<butter>}{<bread>}
  \newcommand*\wrap[2]{#2#1#2}
\makeatother

% Rotate arbitrary pages, even from inside a sideways float.
% Needs {pdflscape} and {bophook} (though both could be replaced rather easily if necessary).
%\makeatletter
%  \AtBeginPage{%
%    \@maybe@rotate@current@page
%  }
%  \def\@maybe@rotate@current@page{%
%    \begingroup
%      \@check@if@current@page@pdflscape
%      \if@tempswa
%        \PLS@Rotate{90}%
%      \else
%        \PLS@RemoveRotate
%      \fi
%    \endgroup
%  }
%  \newcount\c@pdf@rot@mark@count
%  \def\@check@if@current@page@pdflscape{\@tempswafalse}%
%  \def\thispagepdflscape{%
%    \global\advance\c@pdf@rot@mark@count 1\relax
%    \label{@pdf@rot@mark:\the\c@pdf@rot@mark@count}%
%    \xappto\@check@if@current@page@pdflscape{%
%      \noexpand\@check@if@ref@is@current@page@{@pdf@rot@mark:\the\c@pdf@rot@mark@count}%
%    }%
%  }
%  \def\@check@if@ref@is@current@page@#1{%
%    \ifnum\arabic{page}=\@expandable@pageref{#1}\relax
%      \@tempswatrue
%    \fi
%  }
%  \def\@expandable@pageref#1{%
%    \expandafter\@@expandable@pageref\csname r@#1\endcsname
%  }
%  \def\@@expandable@pageref#1{%
%    \ifx#1\relax
%      0%
%    \else
%      \expandafter\@secondofarb#1\@nil
%    \fi
%  }
%  \def\@secondofarb#1#2#3\@nil{#2}%
%\makeatother


%% document elements

\makeatletter
  % taken from scrreprt.sty and modified
  \renewcommand*\maketitle[1][1]{%
    \expandafter\ifnum \csname scr@v@3.12\endcsname>\scr@compatibility\relax
    \else
      \def\and{%
        \end{tabular}%
        \hskip 1em \@plus.17fil%
        \begin{tabular}[t]{c}%
      }%
    \fi
    \if@titlepage
      \begin{titlepage}
        \setcounter{page}{%
          #1%
        }%
        \if@titlepageiscoverpage
          \edef\titlepage@restore{%
            \noexpand\endgroup
            \noexpand\global\noexpand\@colht\the\@colht
            \noexpand\global\noexpand\@colroom\the\@colroom
            \noexpand\global\vsize\the\vsize
            \noexpand\global\noexpand\@titlepageiscoverpagefalse
            \noexpand\let\noexpand\titlepage@restore\noexpand\relax
          }%
          \begingroup
          \topmargin=\dimexpr \coverpagetopmargin-1in\relax
          \oddsidemargin=\dimexpr \coverpageleftmargin-1in\relax
          \evensidemargin=\dimexpr \coverpageleftmargin-1in\relax
          \textwidth=\dimexpr
          \paperwidth-\coverpageleftmargin-\coverpagerightmargin\relax
          \textheight=\dimexpr
          \paperheight-\coverpagetopmargin-\coverpagebottommargin\relax
          \headheight=0pt
          \headsep=0pt
          \footskip=\baselineskip
          \@colht=\textheight
          \@colroom=\textheight
          \vsize=\textheight
          \columnwidth=\textwidth
          \hsize=\columnwidth
          \linewidth=\hsize
        \else
          \let\titlepage@restore\relax
        \fi
        \let\footnotesize\small
        \let\footnoterule\relax
        \let\footnote\thanks
        \renewcommand*\thefootnote{\@fnsymbol\c@footnote}%
        \let\@oldmakefnmark\@makefnmark
        \renewcommand*{\@makefnmark}{\rlap\@oldmakefnmark}%
        \ifx\@extratitle\@empty
          \ifx\@frontispiece\@empty
          \else
            \if@twoside\mbox{}\next@tpage\fi
            \noindent\@frontispiece\next@tdpage
          \fi
        \else
          \noindent\@extratitle
          \ifx\@frontispiece\@empty
          \else
            \next@tpage
            \noindent\@frontispiece
          \fi
          \next@tdpage
        \fi
        \setparsizes{\z@}{\z@}{\z@\@plus 1fil}\par@updaterelative
        \@maketitlepage
        \if@twoside
          \@tempswatrue
          \expandafter\ifnum \@nameuse{scr@v@3.12}>\scr@compatibility\relax
          \else
            \ifx\@uppertitleback\@empty\ifx\@lowertitleback\@empty
              \@tempswafalse
            \fi\fi
          \fi
          \if@tempswa
            \next@tpage
            \begin{minipage}[t]{\textwidth}
              \@uppertitleback
            \end{minipage}\par
            \vfill
            \begin{minipage}[b]{\textwidth}
              \@lowertitleback
            \end{minipage}\par
            \@thanks\let\@thanks\@empty
          \fi
        \fi
        \ifx\@dedication\@empty
        \else
          \next@tdpage\null\vfill
          {\centering\usekomafont{dedication}{\@dedication \par}}%
          \vskip \z@ \@plus3fill
          \@thanks\let\@thanks\@empty
          \cleardoubleemptypage
        \fi
        \ifx\titlepage@restore\relax\else\clearpage\titlepage@restore\fi
      \end{titlepage}
    \else
      \par
      \@tempcnta=%
      #1%
      \relax\ifnum\@tempcnta=1\else
        \ClassWarning{\KOMAClassName}{%
          Optional argument of \string\maketitle\space ignored
          at\MessageBreak
          notitlepage-mode%
        }%
      \fi
      \begingroup
        \let\titlepage@restore\relax
        \renewcommand*\thefootnote{\@fnsymbol\c@footnote}%
        \let\@oldmakefnmark\@makefnmark
        \renewcommand*{\@makefnmark}{\rlap\@oldmakefnmark}%
        \next@tdpage
        \if@twocolumn
          \ifnum \col@number=\@ne
            \ifx\@extratitle\@empty
              \ifx\@frontispiece\@empty\else\if@twoside\mbox{}\fi\fi
            \else
              \@makeextratitle
            \fi
            \ifx\@frontispiece\@empty
              \ifx\@extratitle\@empty\else\next@tdpage\fi
            \else
              \next@tpage
              \@makefrontispiece
              \next@tdpage
            \fi
            \@maketitle
          \else
            \ifx\@extratitle\@empty
              \ifx\@frontispiece\@empty\else\if@twoside\mbox{}\fi\fi
            \else
              \twocolumn[\@makeextratitle]%
            \fi
            \ifx\@frontispiece\@empty
              \ifx\@extratitle\@empty\else\next@tdpage\fi
            \else
              \next@tpage
              \twocolumn[\@makefrontispiece]%
              \next@tdpage
            \fi
            \twocolumn[\@maketitle]%
          \fi
        \else
          \ifx\@extratitle\@empty
            \ifx\@frontispiece\@empty\else \mbox{}\fi
          \else
            \@makeextratitle
          \fi
          \ifx\@frontispiece\@empty
            \ifx\@extratitle\@empty\else\next@tdpage\fi
          \else
            \next@tpage
            \@makefrontispiece
            \next@tdpage
          \fi
          \@maketitle
        \fi
        \ifx\titlepagestyle\@empty\else\thispagestyle{\titlepagestyle}\fi
        \@thanks
      \endgroup
    \fi
    \setcounter{footnote}{0}%
    \expandafter\ifnum \csname scr@v@3.12\endcsname>\scr@compatibility\relax
      \let\thanks\relax
      \let\maketitle\relax
      \let\@maketitle\relax
      \global\let\@thanks\@empty
      \global\let\@author\@empty
      \global\let\@date\@empty
      \global\let\@title\@empty
      \global\let\@subtitle\@empty
      \global\let\@extratitle\@empty
      \global\let\@frontispiece\@empty
      \global\let\@titlehead\@empty
      \global\let\@subject\@empty
      \global\let\@publishers\@empty
      \global\let\@uppertitleback\@empty
      \global\let\@lowertitleback\@empty
      \global\let\@dedication\@empty
      \global\let\author\relax
      \global\let\title\relax
      \global\let\extratitle\relax
      \global\let\titlehead\relax
      \global\let\subject\relax
      \global\let\publishers\relax
      \global\let\uppertitleback\relax
      \global\let\lowertitleback\relax
      \global\let\dedication\relax
      \global\let\date\relax
    \fi
    \global\let\and\relax
  }
  \newcommand*\@maketitlepage{%
    \ifx\@titlehead\@empty \else
      \begin{minipage}[t]{\textwidth}%
        \usekomafont{titlehead}{\@titlehead\par}%
      \end{minipage}\par
    \fi
    \null\vfill
    \begin{center}
      \ifx\@institution\@empty \else
        {\usekomafont{institution}{\@institution\par}}%
        \vskip 1em
      \fi
      \begin{minipage}{\textwidth}\centering
        \rule{\textwidth}{1pt}
        \vskip 2em
        \ifx\@thesistype\@empty \else
          {\usekomafont{thesistype}{\@thesistype\par}}%
          \vskip 3em
        \fi
        {\usekomafont{title}{\huge \@title\par}}%
        \vskip 1em
        {\ifx\@subtitle\@empty\else\usekomafont{subtitle}{\@subtitle\par}\fi}%
        \vskip 2em
        {%
          \usekomafont{author}{%
            \lineskip 0.75em
            \begin{tabular}[t]{c}
              \@author
            \end{tabular}\par
          }%
        }%
        \vskip .5em
        {\usekomafont{date}{\@date\par}}%
        \vskip 2em
        \rule{\textwidth}{1pt}
      \end{minipage}
      \vskip 6em
      {\usekomafont{advisor}{\@advisortext\\\@advisor\par}}%
      \vskip 3em
    \end{center}\par
    \vfill
    \ifx\@titlelower\@empty \else
      \begin{minipage}{\textwidth}%
        {\usekomafont{titlelower}{\@titlelower\par}}%
      \end{minipage}%
    \fi
    \vfill\null
    \ifx\@titlefoot\@empty \else
      \begin{minipage}[b]{\textwidth}%
        \usekomafont{titlefoot}{\@titlefoot\par}%
      \end{minipage}\par
    \fi
  }
  \newcommand*\@adddocumentparameter[2]{%
    \expandafter\def\csname #1\endcsname##1{\expandafter\gdef\csname @#1\endcsname{##1}}%
    \expandafter\def\csname @#1\endcsname{}%
    \newkomafont{#1}{#2}%
  }
  \@adddocumentparameter{institution}{\normalfont\scshape\large}
  \@adddocumentparameter{thesistype}{\normalfont\normalcolor\bfseries\Large}
  \@adddocumentparameter{advisor}{\large}
  \@adddocumentparameter{titlelower}{}
  \@adddocumentparameter{titlefoot}{}
  \setkomafont{date}{\large}
  \newcommand*\@advisortext{betreut durch}
\makeatother

\makeatletter
  \AtBeginDocument{\begingroup\def\and{, }%
    \hypersetup{
      pdftitle=\@title,
      pdfsubject=\@thesistype,
      pdfauthor=\@author,
      pdfcreator={\LaTeX, was sonst?},
    }%
  \endgroup}
  \hypersetup{
    pdfdisplaydoctitle=true,
    pdflang=de-DE,
    pdfduplex=DuplexFlipLongEdge,
    pdfpagelayout=TwoColumnRight,
  }
\makeatother

\makeatletter
  \newcommand*\signaturelinewidth{5cm}
  \newcommand*\signaturelineheight{.4pt}
  \newcommand*\signaturelineraise{.4ex}
  \newcommand*\signaturelabelindent{.5cm}
  \newcommand*\signaturetextindent{}
  \NewDocumentCommand\signatureline{
    O{\signaturelinewidth}    % line width
    O{\signaturelabelindent}  % label margin
    O{%                         text indentation
      \ifx\@empty\signaturetextindent
        .5\dimexpr #2\relax
      \else
        \signaturetextindent
      \fi
    }
    m                         % label
    O{\signaturelineheight}   % line height
    O{\signaturelineraise}    % line raise
    D||{}                     % text
  }{%
    \parbox[t]{#1}{%
      \leftskip #3%
      \mbox{\strut #7}%
      \vskip -#6%
      \hrule height #5%
      \vskip #6%
      \scriptsize
      \leftskip #2%
      \rightskip #2%
      \strut #4%
    }%
  }
  \newcommand*\makeplagiarismstatement{%
    \cleardoublepage
    \thispagestyle{empty}%
    \leavevmode\vfill\noindent\@plagiarismstatement
    \par
    \vskip 1cm
    \hfill
    \signatureline{Ort, Datum}|\@handinlocation, \@handindate|
    \quad
    \signatureline{\@author}
  }
  \@adddocumentparameter{plagiarismstatement}{}%
  \@adddocumentparameter{handinlocation}{}
  \@adddocumentparameter{handindate}{}
\makeatother

\makeatletter
  \newcommand*\makedocumentstart{%
    \cleardoubleoddpage
    \pagenumbering{roman}%
    \maketitle
    \cleardoublepage
    \pdfbookmark[chapter]{\contentsname}{chapter.toc}%
    \tableofcontents
    \cleardoubleevenpage
    \cleardoubleoddpage
    \pagenumbering{arabic}%
  }
  \newcommand*\makedocumentend{%
    \cleardoublepage
    \pdfbookmark[chapter]{\bibname}{chapter.bib}
    \printbibliography
    \cleardoubleevenpage
    \cleardoubleoddpage
    \makeplagiarismstatement
  }
\makeatother

% Needs to be done before loading {scrlayer-scrpage}
%\pagestyle{headings}

\makeatletter
  \setkomafont{caption}{\normalfont\small}
  \setkomafont{captionlabel}{\bfseries\sffamily}
  % Does not work with {caption}, which is used for subfloats.
  % We set the option in the {caption} options instead.
  %\setcapdynwidth{.9\linewidth}
  \captionsetup{margin=.05\textwidth}
  \captionsetup[sub]{labelformat=simple}
  \renewcommand*\thesubfigure{(\alph{subfigure})}
  \renewcommand*\p@subfigure{\thefigure\,}
\makeatother

%\makeatletter
%  \newcommand*\nsubcaption[1]{\setcounter{sub\@captype}{\numexpr(#1)-1\relax}\subcaption}
%  % Create a command that tags every subfloat without a description.
%  \newcommand*\subfloattagpadding{}%
%  \DeclareCaptionFormat{leftbehind}{\raisebox{-\baselineskip}[0pt][0pt]{\subfloattagpadding#1\subfloattagpadding}\hfill\par}
%  \DeclareCaptionFormat{rightbehind}{\hfill\raisebox{-\baselineskip}[0pt][0pt]{\subfloattagpadding#1\subfloattagpadding}\par}
%  \DeclareCaptionFormat{leftoutside}{\raisebox{-\baselineskip}[0pt][0pt]{\llap{\subfloattagpadding#1\subfloattagpadding}}\hfill\par}
%  \DeclareCaptionFormat{rightoutside}{\hfill\raisebox{-\baselineskip}[0pt][0pt]{\rlap{\subfloattagpadding#1\subfloattagpadding}}\par}
%  \newcommand*\subfloatcaptionsetup[1]{%
%    \captionsetup[sub\@captype]{#1}%
%  }
%  \NewDocumentCommand\tagsubfloats{ o }{%
%    \subfloatcaptionsetup{singlelinecheck=false, skip=0pt}%
%    \IfValueT{#1}{\subfloatcaptionsetup{format=#1}}%
%    \apptocmd\subcaption@iiminipage{\caption{}}{}{}%
%  }
%\makeatother



%%% miscellaneous

\institution{%
  Eberhard-Karls-Universität Tübingen\\%
  Physikalisches Institut\quad\(\circ\)\quad
  Experimentalphysik~II%
}
\titlelower{%
  \centering
  \includegraphics[width=.4\textwidth]{pit2logo_greyscale}%
}
\plagiarismstatement{%
  Ich versichere, dass ich diese Arbeit selbstständig verfasst und keine anderen als die angegebenen Hilfsmittel und Quellen benutzt habe.
  Alle wörtlich oder sinngemäß aus anderen Werken übernommenen Aussagen sind als solche gekennzeichnet.
  Die Arbeit ist weder vollständig noch in wesentlichen Teilen Gegenstand eines anderen Prüfungsverfahrens gewesen und ich habe sie weder vollständig noch in wesentlichen Teilen bereits veröffentlicht.
  Das in Dateiform eingereichte Exemplar stimmt mit eingereichten gebundenen Exemplaren überein.%
}


%\DeclareSIUnit\dBm{dBm}
%\DeclareSIUnit\Phizero{\Phi_0}



% Provide a switch for the size reduced version of this document.
\makeatletter
  \newif\ifreducesize
%  \tikzset{
%    % Make sure this key exists.
%    install size reduction options/.append code={},
%    % Provide a switch handler, first argument is proper value, second is size reduced value.
%    /handlers/.depending on size reduction/.code 2 args={%
%      \ifreducesize
%        \pgfkeys{\pgfkeyscurrentpath={#2}}%
%      \else
%        \pgfkeys{\pgfkeyscurrentpath={#1}}%
%      \fi
%    },
%  }
  \AtBeginDocument{%
    \ifreducesize
      \renewcommand*\makeplagiarismstatement{%
         \cleardoublepage
         \thispagestyle{empty}%
         \leavevmode\vfill\noindent
         Zur Reduktion der Dateigröße wurden in der vorliegenden Version dieser Arbeit modifizierte Abbildungen verwendet.
         Sie weicht in dieser Hinsicht von der eingereichten Version ab.
       }%
      \graphicspath{{./img/common/}{./img/sd/}}%
      \svgpath{{./img/common/}{./img/sd/}}%
      \svgsetup{inkscapepath={./svg-inkscape-sd/}}%
%      \tikzset{install size reduction options}%
    \else
      \graphicspath{{./img/common/}{./img/hd/}}%
      \svgpath{{./img/common/}{./img/hd/}}%
      \svgsetup{inkscapepath={./svg-inkscape-hd/}}%
    \fi
  }
\makeatother

\makeatletter
  \newif\ifshowtodos
  \newcommand*\@show@todo[1]{%
    \GenericWarning{(todo)\@spaces}{%
      LaTeX Warning:
      TODO: #1%
    }%
  }
  \newcommand*\todo[1]{%
    \ifshowtodos
      \@show@todo{#1}%
    \fi
    \ignorespaces
  }
\makeatother
